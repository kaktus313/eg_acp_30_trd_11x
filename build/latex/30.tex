% Generated by Sphinx.
\def\sphinxdocclass{report}
\newif\ifsphinxKeepOldNames \sphinxKeepOldNamestrue
\documentclass[letterpaper,10pt,russian]{sphinxmanual}
\usepackage{iftex}

\ifPDFTeX
  \usepackage[utf8]{inputenc}
\fi
\ifdefined\DeclareUnicodeCharacter
  \DeclareUnicodeCharacter{00A0}{\nobreakspace}
\fi
\usepackage{cmap}
\usepackage{fontspec}
\usepackage{amsmath,amssymb,amstext}
\usepackage{babel}
\setmainfont{DejaVu Serif}
\setsansfont{DejaVu Sans}
\setmonofont{DejaVu Sans Mono}
\usepackage[Bjornstrup]{fncychap}
\usepackage{longtable}
\usepackage{sphinx}
\usepackage{multirow}
\usepackage{eqparbox}


\addto\captionsrussian{\renewcommand{\figurename}{Рис.\@ }}
\addto\captionsrussian{\renewcommand{\tablename}{Таблица }}
\SetupFloatingEnvironment{literal-block}{name=Список }

\addto\extrasrussian{\def\pageautorefname{page}}

\setcounter{tocdepth}{1}
\usepackage[titles]{tocloft}
\cftsetpnumwidth {1.25cm}\cftsetrmarg{1.5cm}
\setlength{\cftchapnumwidth}{0.75cm}
\setlength{\cftsecindent}{\cftchapnumwidth}
\setlength{\cftsecnumwidth}{1.25cm}

\title{Документация к Модулю 1С-ЕГАИС для Управление торговлей 11.2 Documentation}
\date{мая 16, 2017}
\release{1.0}
\author{Романов Илья}
\newcommand{\sphinxlogo}{}
\renewcommand{\releasename}{Выпуск}
\makeindex

\makeatletter
\def\PYG@reset{\let\PYG@it=\relax \let\PYG@bf=\relax%
    \let\PYG@ul=\relax \let\PYG@tc=\relax%
    \let\PYG@bc=\relax \let\PYG@ff=\relax}
\def\PYG@tok#1{\csname PYG@tok@#1\endcsname}
\def\PYG@toks#1+{\ifx\relax#1\empty\else%
    \PYG@tok{#1}\expandafter\PYG@toks\fi}
\def\PYG@do#1{\PYG@bc{\PYG@tc{\PYG@ul{%
    \PYG@it{\PYG@bf{\PYG@ff{#1}}}}}}}
\def\PYG#1#2{\PYG@reset\PYG@toks#1+\relax+\PYG@do{#2}}

\expandafter\def\csname PYG@tok@nv\endcsname{\def\PYG@tc##1{\textcolor[rgb]{0.73,0.38,0.84}{##1}}}
\expandafter\def\csname PYG@tok@nn\endcsname{\let\PYG@bf=\textbf\def\PYG@tc##1{\textcolor[rgb]{0.05,0.52,0.71}{##1}}}
\expandafter\def\csname PYG@tok@cpf\endcsname{\let\PYG@it=\textit\def\PYG@tc##1{\textcolor[rgb]{0.25,0.50,0.56}{##1}}}
\expandafter\def\csname PYG@tok@cp\endcsname{\def\PYG@tc##1{\textcolor[rgb]{0.00,0.44,0.13}{##1}}}
\expandafter\def\csname PYG@tok@nt\endcsname{\let\PYG@bf=\textbf\def\PYG@tc##1{\textcolor[rgb]{0.02,0.16,0.45}{##1}}}
\expandafter\def\csname PYG@tok@mi\endcsname{\def\PYG@tc##1{\textcolor[rgb]{0.13,0.50,0.31}{##1}}}
\expandafter\def\csname PYG@tok@mf\endcsname{\def\PYG@tc##1{\textcolor[rgb]{0.13,0.50,0.31}{##1}}}
\expandafter\def\csname PYG@tok@gt\endcsname{\def\PYG@tc##1{\textcolor[rgb]{0.00,0.27,0.87}{##1}}}
\expandafter\def\csname PYG@tok@gr\endcsname{\def\PYG@tc##1{\textcolor[rgb]{1.00,0.00,0.00}{##1}}}
\expandafter\def\csname PYG@tok@sh\endcsname{\def\PYG@tc##1{\textcolor[rgb]{0.25,0.44,0.63}{##1}}}
\expandafter\def\csname PYG@tok@s2\endcsname{\def\PYG@tc##1{\textcolor[rgb]{0.25,0.44,0.63}{##1}}}
\expandafter\def\csname PYG@tok@mo\endcsname{\def\PYG@tc##1{\textcolor[rgb]{0.13,0.50,0.31}{##1}}}
\expandafter\def\csname PYG@tok@vc\endcsname{\def\PYG@tc##1{\textcolor[rgb]{0.73,0.38,0.84}{##1}}}
\expandafter\def\csname PYG@tok@gs\endcsname{\let\PYG@bf=\textbf}
\expandafter\def\csname PYG@tok@c\endcsname{\let\PYG@it=\textit\def\PYG@tc##1{\textcolor[rgb]{0.25,0.50,0.56}{##1}}}
\expandafter\def\csname PYG@tok@kp\endcsname{\def\PYG@tc##1{\textcolor[rgb]{0.00,0.44,0.13}{##1}}}
\expandafter\def\csname PYG@tok@go\endcsname{\def\PYG@tc##1{\textcolor[rgb]{0.20,0.20,0.20}{##1}}}
\expandafter\def\csname PYG@tok@kt\endcsname{\def\PYG@tc##1{\textcolor[rgb]{0.56,0.13,0.00}{##1}}}
\expandafter\def\csname PYG@tok@ne\endcsname{\def\PYG@tc##1{\textcolor[rgb]{0.00,0.44,0.13}{##1}}}
\expandafter\def\csname PYG@tok@err\endcsname{\def\PYG@bc##1{\setlength{\fboxsep}{0pt}\fcolorbox[rgb]{1.00,0.00,0.00}{1,1,1}{\strut ##1}}}
\expandafter\def\csname PYG@tok@kd\endcsname{\let\PYG@bf=\textbf\def\PYG@tc##1{\textcolor[rgb]{0.00,0.44,0.13}{##1}}}
\expandafter\def\csname PYG@tok@nc\endcsname{\let\PYG@bf=\textbf\def\PYG@tc##1{\textcolor[rgb]{0.05,0.52,0.71}{##1}}}
\expandafter\def\csname PYG@tok@no\endcsname{\def\PYG@tc##1{\textcolor[rgb]{0.38,0.68,0.84}{##1}}}
\expandafter\def\csname PYG@tok@vg\endcsname{\def\PYG@tc##1{\textcolor[rgb]{0.73,0.38,0.84}{##1}}}
\expandafter\def\csname PYG@tok@ow\endcsname{\let\PYG@bf=\textbf\def\PYG@tc##1{\textcolor[rgb]{0.00,0.44,0.13}{##1}}}
\expandafter\def\csname PYG@tok@ni\endcsname{\let\PYG@bf=\textbf\def\PYG@tc##1{\textcolor[rgb]{0.84,0.33,0.22}{##1}}}
\expandafter\def\csname PYG@tok@nd\endcsname{\let\PYG@bf=\textbf\def\PYG@tc##1{\textcolor[rgb]{0.33,0.33,0.33}{##1}}}
\expandafter\def\csname PYG@tok@kn\endcsname{\let\PYG@bf=\textbf\def\PYG@tc##1{\textcolor[rgb]{0.00,0.44,0.13}{##1}}}
\expandafter\def\csname PYG@tok@ss\endcsname{\def\PYG@tc##1{\textcolor[rgb]{0.32,0.47,0.09}{##1}}}
\expandafter\def\csname PYG@tok@sc\endcsname{\def\PYG@tc##1{\textcolor[rgb]{0.25,0.44,0.63}{##1}}}
\expandafter\def\csname PYG@tok@na\endcsname{\def\PYG@tc##1{\textcolor[rgb]{0.25,0.44,0.63}{##1}}}
\expandafter\def\csname PYG@tok@gu\endcsname{\let\PYG@bf=\textbf\def\PYG@tc##1{\textcolor[rgb]{0.50,0.00,0.50}{##1}}}
\expandafter\def\csname PYG@tok@il\endcsname{\def\PYG@tc##1{\textcolor[rgb]{0.13,0.50,0.31}{##1}}}
\expandafter\def\csname PYG@tok@nl\endcsname{\let\PYG@bf=\textbf\def\PYG@tc##1{\textcolor[rgb]{0.00,0.13,0.44}{##1}}}
\expandafter\def\csname PYG@tok@sr\endcsname{\def\PYG@tc##1{\textcolor[rgb]{0.14,0.33,0.53}{##1}}}
\expandafter\def\csname PYG@tok@k\endcsname{\let\PYG@bf=\textbf\def\PYG@tc##1{\textcolor[rgb]{0.00,0.44,0.13}{##1}}}
\expandafter\def\csname PYG@tok@o\endcsname{\def\PYG@tc##1{\textcolor[rgb]{0.40,0.40,0.40}{##1}}}
\expandafter\def\csname PYG@tok@gh\endcsname{\let\PYG@bf=\textbf\def\PYG@tc##1{\textcolor[rgb]{0.00,0.00,0.50}{##1}}}
\expandafter\def\csname PYG@tok@cm\endcsname{\let\PYG@it=\textit\def\PYG@tc##1{\textcolor[rgb]{0.25,0.50,0.56}{##1}}}
\expandafter\def\csname PYG@tok@gd\endcsname{\def\PYG@tc##1{\textcolor[rgb]{0.63,0.00,0.00}{##1}}}
\expandafter\def\csname PYG@tok@se\endcsname{\let\PYG@bf=\textbf\def\PYG@tc##1{\textcolor[rgb]{0.25,0.44,0.63}{##1}}}
\expandafter\def\csname PYG@tok@mh\endcsname{\def\PYG@tc##1{\textcolor[rgb]{0.13,0.50,0.31}{##1}}}
\expandafter\def\csname PYG@tok@kc\endcsname{\let\PYG@bf=\textbf\def\PYG@tc##1{\textcolor[rgb]{0.00,0.44,0.13}{##1}}}
\expandafter\def\csname PYG@tok@c1\endcsname{\let\PYG@it=\textit\def\PYG@tc##1{\textcolor[rgb]{0.25,0.50,0.56}{##1}}}
\expandafter\def\csname PYG@tok@nb\endcsname{\def\PYG@tc##1{\textcolor[rgb]{0.00,0.44,0.13}{##1}}}
\expandafter\def\csname PYG@tok@sd\endcsname{\let\PYG@it=\textit\def\PYG@tc##1{\textcolor[rgb]{0.25,0.44,0.63}{##1}}}
\expandafter\def\csname PYG@tok@vi\endcsname{\def\PYG@tc##1{\textcolor[rgb]{0.73,0.38,0.84}{##1}}}
\expandafter\def\csname PYG@tok@s1\endcsname{\def\PYG@tc##1{\textcolor[rgb]{0.25,0.44,0.63}{##1}}}
\expandafter\def\csname PYG@tok@cs\endcsname{\def\PYG@tc##1{\textcolor[rgb]{0.25,0.50,0.56}{##1}}\def\PYG@bc##1{\setlength{\fboxsep}{0pt}\colorbox[rgb]{1.00,0.94,0.94}{\strut ##1}}}
\expandafter\def\csname PYG@tok@s\endcsname{\def\PYG@tc##1{\textcolor[rgb]{0.25,0.44,0.63}{##1}}}
\expandafter\def\csname PYG@tok@gi\endcsname{\def\PYG@tc##1{\textcolor[rgb]{0.00,0.63,0.00}{##1}}}
\expandafter\def\csname PYG@tok@kr\endcsname{\let\PYG@bf=\textbf\def\PYG@tc##1{\textcolor[rgb]{0.00,0.44,0.13}{##1}}}
\expandafter\def\csname PYG@tok@mb\endcsname{\def\PYG@tc##1{\textcolor[rgb]{0.13,0.50,0.31}{##1}}}
\expandafter\def\csname PYG@tok@w\endcsname{\def\PYG@tc##1{\textcolor[rgb]{0.73,0.73,0.73}{##1}}}
\expandafter\def\csname PYG@tok@ch\endcsname{\let\PYG@it=\textit\def\PYG@tc##1{\textcolor[rgb]{0.25,0.50,0.56}{##1}}}
\expandafter\def\csname PYG@tok@sb\endcsname{\def\PYG@tc##1{\textcolor[rgb]{0.25,0.44,0.63}{##1}}}
\expandafter\def\csname PYG@tok@ge\endcsname{\let\PYG@it=\textit}
\expandafter\def\csname PYG@tok@m\endcsname{\def\PYG@tc##1{\textcolor[rgb]{0.13,0.50,0.31}{##1}}}
\expandafter\def\csname PYG@tok@sx\endcsname{\def\PYG@tc##1{\textcolor[rgb]{0.78,0.36,0.04}{##1}}}
\expandafter\def\csname PYG@tok@si\endcsname{\let\PYG@it=\textit\def\PYG@tc##1{\textcolor[rgb]{0.44,0.63,0.82}{##1}}}
\expandafter\def\csname PYG@tok@gp\endcsname{\let\PYG@bf=\textbf\def\PYG@tc##1{\textcolor[rgb]{0.78,0.36,0.04}{##1}}}
\expandafter\def\csname PYG@tok@bp\endcsname{\def\PYG@tc##1{\textcolor[rgb]{0.00,0.44,0.13}{##1}}}
\expandafter\def\csname PYG@tok@nf\endcsname{\def\PYG@tc##1{\textcolor[rgb]{0.02,0.16,0.49}{##1}}}

\def\PYGZbs{\char`\\}
\def\PYGZus{\char`\_}
\def\PYGZob{\char`\{}
\def\PYGZcb{\char`\}}
\def\PYGZca{\char`\^}
\def\PYGZam{\char`\&}
\def\PYGZlt{\char`\<}
\def\PYGZgt{\char`\>}
\def\PYGZsh{\char`\#}
\def\PYGZpc{\char`\%}
\def\PYGZdl{\char`\$}
\def\PYGZhy{\char`\-}
\def\PYGZsq{\char`\'}
\def\PYGZdq{\char`\"}
\def\PYGZti{\char`\~}
% for compatibility with earlier versions
\def\PYGZat{@}
\def\PYGZlb{[}
\def\PYGZrb{]}
\makeatother

\renewcommand\PYGZsq{\textquotesingle}

\begin{document}

\maketitle
\tableofcontents
\phantomsection\label{index::doc}


Содержание:


\chapter{О продукте}
\label{intro::doc}\label{intro:id1}
Модуль 1С-ЕГАИС - это прикладной программный продукт, предоставляющий пользователю полный набор функциональных возможностей для эффективной работы с \textsc{ЕГАИС} (Единая государственная автоматизированная информационная система).

\begin{notice}{hint}{Подсказка:}
Единая государственная автоматизированная информационная система (ЕГАИС) — автоматизированная система, предназначенная для государственного контроля над объёмом производства и оборота этилового спирта, алкогольной и спиртосодержащей продукции.
\end{notice}

Модуль 1С-ЕГАИС не является самостоятельным программным продуктом и может быть использован в качестве дополнения к определенным типовым конфигурациям 1С:Предприятие 8. Подробнее см. список поддерживаемых конфигураций.

Данное руководство охватывает функциональные возможности ПРОФ версий модуля. Описание версий продукта см. в разделе ``Варианты поставки''.


\section{Поддерживаемые конфигурации 1С:Предприятие}
\label{intro:id2}
Модуль 1С-ЕГАИС может быть установлен и использован совместно со следующими типовыми конфигурациями фирмы 1С:
\begin{itemize}
\item {} 
1С:Бухгалтерия предприятия 8'', ред 3.0

\item {} 
1С:Управление торговлей 8, ред 10.3

\item {} 
1С:Управление торговлей 8, ред 11.1 и выше

\end{itemize}

\begin{notice}{important}{Важно:}
Базовые версии модуля могут быть установлены на любые версии(базовые или ПРОФ поставки) выше перечисленных конфигураций 1С. Модуль версии ПРОФ совместим только с ПРОФ поставками т.е. не может быть установлен, например, на базовую версию программы ``1С:Бухгатерия 8''
\end{notice}


\section{Варианты поставки}
\label{intro:id3}
Существует два варианта поставки модуля 1С-ЕГАИС:
\begin{enumerate}
\item {} 
Базовая версия - реализована в виде внешней обработки 1С, совместима с базовыми и ПРОФ версиями 1С.

\item {} 
ПРОФ (или профессиональная)версия - реализована в виде дополнения конфигурации. Не изменяет штатные объекты метаданных конфигурации 1С, а только дополняет конфигурацию объектами метаданных Модуля 1С-ЕГАИС.

\end{enumerate}


\section{Функциональные возможности}
\label{intro:id4}

\section{Лицензирование}
\label{intro:id5}\begin{itemize}
\item {} 
Программа лицензируется на каждую организацию, от лица которой ведется деятельность в ЕГАИС с использованием Модуля 1С-ЕГАИС. На каждое дополнительное обособленное подразделение необходимо приобретение дополнительной лицензии. Для индивидуальных предпринимателей покупка дополнительных лицензий не требуется.

\item {} 
Программа поставляется с открытым кодом. Изменения в коде программы допускаются, однако в этом случае, по усмотрению разработчика, могут быть пересмотрены условия предоставления как льготной, так и платной технической поддержки

\end{itemize}


\section{Условия предоставления технической поддержки}
\label{intro:id6}\begin{itemize}
\item {} 
Техническая поддержка предоставляется на лицензии зарегистрированные в базе данных разработчика. При этом разработчик, вправе запросить у клиента информацию однозначно индетифицирующую клиента в базе данных разработчика. Это может быть номер заказа, контактное лицо на которое оформлялся заказ, ИНН организации и т.д.

\item {} 
Техподдержка осуществляется с использованием средств удаленного сопровождения TeamViewer 8. При необходимости программа предоставляется  пользователю. В стоимость приобретения обработки включена техническая поддержка сроком 3 месяца с момента приобретения. Дальнейшая подержка осуществляется на возмездной основе в соответствии с тарифным планом, выбранным пользователем, см. далее.

\item {} 
Тарифный план ``БАЗОВЫЙ''. Рассчитан на пользователей с потребностями в регулярном получении обновлений, а также в оперативном исправлении критических ошибок в режиме исполнения продукта в зоне ответственности разработчика.

\item {} 
Тарифный план ``МАКСИМАЛЬНЫЙ''. Комплекс услуг, включающий в себя: доступ к системе автоматических обновлений, дистанционное консультирование в области применения продукта (телефон, skype, viber, email), решение технических вопросов возникающих в ходе эксплуатации продукта, за исключением вопросов, находящихся в зоне ответственности системного администратора компании пользователя.

\item {} 
При первом приобретении, в поставку продукта включается техническая поддержка по тарифному плану ``МАКСИМАЛЬНЫЙ'' сроком на 3 мес. (90 дней), начиная с даты приобретения продукта.

\item {} 
При необходимости доработок под индивидуальные особенности конфигураций 1С клиента, это может быть выполнено на возмездной основе. На усмотрение разработчика некоторая часть стоимости доработок может быть включена в стоимость приобретения за счет льготной техподдержки.

\item {} 
Обновления программы осуществляются автоматически. После покупки вы получите письмо с данными для регистрации на сервере обновлений. Стоимость обновлений включена во все тарифные планы технической поддержки.

\end{itemize}


\chapter{Алкогольная продукция ЕГАИС}
\label{alcprod::doc}\label{alcprod:id1}

\section{Отбор по критериям}
\label{alcprod:id2}
При открытии окна ``Алкогольная продукция ЕГАИС'' вы увидите форму, в которой в левой половине находится список алкогольной продукции из ЕГАИС, а справа список вашей номенклатуры.
\begin{figure}[htbp]
\centering

\noindent\sphinxincludegraphics{{alcprod01}.png}
\end{figure}

Над обоими списками находятся опции для отбора, вы можете как в левом, так и в правом списке отфильтровать номенклатуру по сопоставленности переключая тумблеры ``Все'', ``Связанные'', ``Не связанные''.

Также доступны более детальные отборы по критериям. С обоих сторон доступны галочки ``Отбор по свойствам''. Включая их вы можете отбирать номенклатуру по разным свойствам. Если вы например с левой стороны включите галочку ``Вид продукции'' под галочкой ``Отбор по свойствам'', то в списке номенклатуры ЕГАИС отобразится только та номенклатура, которая соответствует тому коду продукции, которому соответствует номенклатура, выбранная справа. Чтобы поменять выбор, просто щелкните один раз на другую номенклатуру и список, в котором ведётся отбор переформируется.
\begin{figure}[htbp]
\centering

\noindent\sphinxincludegraphics{{alcprod03}.png}
\end{figure}

На некоторых позициях номенклатуры ЕГАИС вы можете увидеть вот такой знак (восклицательный знак в красном треугольнике)
\begin{figure}[htbp]
\centering

\noindent\sphinxincludegraphics{{alcprod02}.png}
\end{figure}

Это значит что у данной номенклатуры не сопоставлен производитель (импортер). Если вы попытаетесь сопоставить эту номенклатуру, выведется сообщение что производитель у этой номенклатуры не сопоставлен и откроется форма сопоставления организаций.


\section{Множественное сопоставление}
\label{alcprod:id3}
Одной номенклатуре ЕГАИС может соответствовать несколько наименований алкогольной продукции из ИБ и вы можете это отразить. Если вы откроете номенклатуру из списка алкогольной продукции ЕГАИС, то вы увидите свойства номенклатуры, в которой в поле ``Номенклатура'' будет одна номенклатура из базы, соответствующая номенклатуре из ИБ.
\begin{figure}[htbp]
\centering

\noindent\sphinxincludegraphics{{alcprod04}.png}
\end{figure}

Она считается основной для данной номенклатуры ЕГАИС. Но также доступна возможность сопоставить и другие. Во вкладке сопоставленная номенклатура вы можете добавить номенклатуру, соответствующую данной алкогольной продукции ЕГАИС, а также сделать одну из позиции номенклатур основной.
\begin{figure}[htbp]
\centering

\noindent\sphinxincludegraphics{{alcprod05}.png}
\end{figure}


\chapter{ТТН ЕГАИС}
\label{ttn::doc}\label{ttn:id1}

\section{Создание документа ТТН}
\label{ttn:id2}
Документ ТТН можно создать на основании документов ``Реализация товаров и услуг'', ``Возврат товаров поставщику'' и ``Перемещение товаров''.
При нажатии кнопки ``Создать'' откроется форма выбора документа основания для новой ТТН.
\begin{figure}[htbp]
\centering

\noindent\sphinxincludegraphics{{ttn11}.png}
\end{figure}

После выбора документа откроется форма нового документа ТТН, заполненного на основании выбранного документа.
\begin{figure}[htbp]
\centering

\noindent\sphinxincludegraphics{{ttn12}.png}
\end{figure}
\begin{figure}[htbp]
\centering

\noindent\sphinxincludegraphics{{ttn13}.png}
\end{figure}

Для отправки ТТН в ЕГАИС нужно заполнить справки Б у каждой номенклатуры. Это можно сделать вручную, либо автоматически по FIFO.
При нажатии на кнопку ``Заполнить справки по FIFO'' справки подберутся автоматически.
\begin{figure}[htbp]
\centering

\noindent\sphinxincludegraphics{{ttn14}.png}
\end{figure}
\begin{figure}[htbp]
\centering

\noindent\sphinxincludegraphics{{ttn15}.png}
\end{figure}

При нажатии на кнопку ``Подобрать справки из остатков'' будет открыт список справок Б, в нем же показаны остатки выбранного товара по этим справкам.
\begin{figure}[htbp]
\centering

\noindent\sphinxincludegraphics{{ttn16}.png}
\end{figure}
\begin{figure}[htbp]
\centering

\noindent\sphinxincludegraphics{{ttn17}.png}
\end{figure}


\section{Сопоставление номенклатуры по товарам из ТТН}
\label{ttn:id3}
В табличной части ``Товары'' документа ТТН есть кнопка ``Сопоставить номенклатуру''.
\begin{figure}[htbp]
\centering

\noindent\sphinxincludegraphics{{ttn01}.png}
\end{figure}

При её нажатии появится окно списка номенклатуры ЕГАИС и ИБ.
\begin{figure}[htbp]
\centering

\noindent\sphinxincludegraphics{{ttn02}.png}
\end{figure}

Если кнопка была нажата из входящего документа ТТН, то отбор по товарам из данной ТТН будет произведен в списке алкогольной продукции ЕГАИС, а если из исходящего, то в списке номенклатуры ИБ.


\section{Выбор номенклатуры из множественного сопоставления}
\label{ttn:id4}
Если номенклатуре ЕГАИС, указанной в табличной части ``Товары'' входящей ТТН сопоставлено несколько номенклатур из ИБ, то вы можете выбрать одну из них.
\begin{figure}[htbp]
\centering

\noindent\sphinxincludegraphics{{ttn03}.png}
\end{figure}


\section{Ручное изменение данных}
\label{ttn:id5}
Документ Товарно-транспортная накладная ЕГАИС в конфигурации существует для отражения состояния соответствующего документа в ЕГАИС. По результатам обмена с УТМ у документов меняется статус, присваивается номер идентификатора, номер и дата фиксации. Но в ряде случаев обмен может работать некорректно, (изменения могут не зафиксироваться например из-за плохого соединения, или неправильной работы УТМ) и если вы уверены что данные документа в вашей базе отличаются от данных того же документа в ЕГАИС, предусмотрена возможность изменить их самому, как бы вручную довести до актуальности.
\begin{figure}[htbp]
\centering

\noindent\sphinxincludegraphics{{ttn04}.png}
\end{figure}

Если бы пользователю была дана возможность редактировать данные свободно, без определенного алгоритма сохранения изменения данных, то ни у пользователей, ни у техподдержки не было бы уверенности в актуальности данных документов в базе.
С помощью ручного изменения данных пользователи могут изменять статус обработки, идентификатор ТТН, дату и номер фиксации в ЕГАИС, однако каждое это изменение будет записываться в регистр Ручные изменения данных ТТН. При включении режима изменения в регистр записываются текущие данные ТТН. После этого их можно изменять, каждое изменение будет фиксироваться. В любой момент можно откатить все изменения вернув первоначальные данные.

\begin{notice}{warning}{Предупреждение:}
Изменение данных в информационной базе не повлечет за собой изменение их в ЕГАИС.
\end{notice}
\begin{figure}[htbp]
\centering

\noindent\sphinxincludegraphics{{ttn05}.png}
\end{figure}

Для возможности изменения данных вручную, нужно включить разрешение ручного изменения нажав на галочку ``Разрешить ручное изменение''. После положительного ответа на предупреждение, изменение станет доступным. Справа отображается список регистра Ручные изменения данных ТТН по текущей ТТН. Когда изменение станет доступным, в списке будет одна запись.
\begin{figure}[htbp]
\centering

\noindent\sphinxincludegraphics{{ttn06}.png}
\end{figure}

После заполнения данных вручную нажимайте кнопку ``Изменить данные'', окно закроется и изменения будут видны в документе.
\begin{figure}[htbp]
\centering

\noindent\sphinxincludegraphics{{ttn07}.png}
\end{figure}
\begin{figure}[htbp]
\centering

\noindent\sphinxincludegraphics{{ttn08}.png}
\end{figure}

Чтобы вернуться к первоначальным данным, которые были до ручного изменения нажмите кнопку ``Вернуть первоначальные данные''.
\begin{figure}[htbp]
\centering

\noindent\sphinxincludegraphics{{ttn09}.png}
\end{figure}
\begin{figure}[htbp]
\centering

\noindent\sphinxincludegraphics{{ttn10}.png}
\end{figure}


\chapter{Организации ЕГАИС}
\label{org::doc}\label{org:id1}

\section{Отбор по критериям}
\label{org:id2}
При открытии окна ``Организации ЕГАИС'' вы увидите форму, в которой в левой половине находится список организаций ЕГАИС, а справа список контрагентов из вашей базы.
\begin{figure}[htbp]
\centering

\noindent\sphinxincludegraphics{{org01}.png}
\end{figure}

Над обоими списками находятся опции для отбора, вы можете как в левом, так и в правом списке отфильтровать организации по сопоставленности переключая тумблеры ``Все'', ``Связанные'', ``Не связанные''.

Также доступны более детальные отборы по критериям. С обоих сторон доступны галочки ``Отбор по свойствам''. Включая их вы можете отбирать организации по разным свойствам. Если вы например с левой стороны включите галочку ``Вид продукции'' под галочкой ``Отбор по свойствам'', то в списке номенклатуры ЕГАИС отобразится только та номенклатура, которая соответствует тому коду продукции, которому соответствует номенклатура, выбранная справа. Чтобы поменять выбор, просто щелкните один раз на другую номенклатуру и список, в котором ведётся отбор переформируется.
\begin{figure}[htbp]
\centering

\noindent\sphinxincludegraphics{{org02}.png}
\end{figure}


\chapter{Протокол обмена ЕГАИС}
\label{exprot::doc}\label{exprot:id1}
Вся работа с ЕГАИС ведётся через протокол обмена.
\begin{figure}[htbp]
\centering

\noindent\sphinxincludegraphics{{exprot01}.png}
\end{figure}

В нем находится вся информация об исходящих и входящих запросах ЕГАИС.
По кнопке ``Обмен с ЕГАИС'' загружаются входящие запросы из УТМ и записываются в протокол, после чего необработанные ответы из протокола сразу же обрабатываются.

Зеленым цветом выделены входящие запросы.


\chapter{Корректировка остатков ЕГАИС}
\label{korost::doc}\label{korost:id1}
Обработка используется для синхронизации остатков между информационной базой и ЕГАИС с учетом остатков в обоих регистрах (оптовом и розничном). Сначала загружаются остатки ЕГАИС, сравниваются с текущими остатками из ИБ, и на основе этих несоответствий создаются документы, после отправки которых вы скорректируете в ЕГАИС информацию по текущим остаткам из информационной базы.

В обработке действует пошаговый интерфейс, вы постранично вводите нужную обработке информацию.
\begin{figure}[htbp]
\centering

\noindent\sphinxincludegraphics{{korost01}.png}
\end{figure}

На первой странице вы выбираете организацию. Также вы можете выбрать группу номенклатуры, которую будете синхронизировать, если включите галочку ``Отбирать по группам'', тогда будет видно поле ``Группа номенклатуры'', в котором вы можете выбрать нужную группу из справочника ``Номенклатура''. При выключенной галочке будет выбрана вся алкогольная продукция.
Переключатель ``Все, Маркируемые, Немаркируемые'' отберёт эту номенклатуру по кодам вида продукции. То есть, если вы выберете, например, ``Немаркируемые'', тогда будет отбираться только номенклатура с кодами 261, 262, 263, 500, 510, 520. При положении переключателя ``Маркируемые'' будет отбираться номенклатура со всеми остальными кодами исключая перечисленные ранее.

При нажатии на кнопку ``Перейти вперед'' вы перейдете на страницу выбора актуальных остатков.
\begin{figure}[htbp]
\centering

\noindent\sphinxincludegraphics{{korost02}.png}
\end{figure}

Вы можете запросить остатки из ЕГАИС или выбрать из запрашиваемых ранее, используя документы ``Синхронизация остатков ЕГАИС'' и ``Загрузка остатков из торгового зала''.

Если хотя бы один переключатель установлен в положение ``Запросить остатки ...'', тогда после нажатия на кнопку ``Перейти вперед'' в ЕГАИС отправляется запрос. Убедитесь в правильности настроек УТМ для связи с ЕГАИС.
\begin{figure}[htbp]
\centering

\noindent\sphinxincludegraphics{{korost03}.png}
\end{figure}

Обработка отправляет запрос и каждые 2 минуты после этого проверяет, не пришел ли ответ. Эта страница не показывается, если остатки из обоих регистров были выбраны из загруженных документов, так как остатки уже загружены и ожидание не требуется. Когда ответ приходит, на экран выводится следующая страница, уведомляющая о том, что остатки загружены и их можно посмотреть.
\begin{figure}[htbp]
\centering

\noindent\sphinxincludegraphics{{korost04}.png}
\end{figure}

Теперь вы можете приступить к автоматическому созданию корректировочных документов нажатием кнопки ``Перейти вперед''.
\begin{figure}[htbp]
\centering

\noindent\sphinxincludegraphics{{korost05}.png}
\end{figure}

В таблицу корректировок, которая видна на экране, записываются сначала запрашиваемые остатки из ЕГАИС или из загруженных ранее документов (в зависимости от того, что вы до этого выбрали). После этого в эту таблицу добавляются остатки из информационной базы. Обратите внимание, что в таблицу могут вывестись номенклатура ЕГАИС, не сопоставленная с номенклатурой ИБ и наоборот, номенклатура из регистра ``Остатки на складе'', которой не соответствует ни одна из позиции номенклатуры ЕГАИС. Такая номенклатура попадает в данную таблицу, даже если она не попадает в отбор по выбранной номенклатурной группе - это сделано для того, чтобы не сопоставленная номенклатура была видна сразу. Вообще если есть несопоставленная номенклатура, остатки не смогут быть скорректированы правильно.

Колонки ``Регистр1'' и ``Регистр2'' заполняются по данным ЕГАИС, колонка ``Остаток на складе'' заполняется по данным информационной базы, остальные две колонки ``Передать в торговый зал'' и ``Постановка / списание'' заполняются из показателей остальных колонок и на их основе делаются корректировочные документы. В поле ``Передать в торговый зал'' просто передается показатель первого регистра из ЕГАИС. Поле ``Постановка / списание'' рассчитывается по формуле:

Постановка / списание = Остаток на складе - Регистр1 - Регистр2

Таким образом, если в таблице есть ненулевые показатели колонки ``Постановка / списание'', на основе этих показателей создается документ ``Передача продукции в торговый зал ЕГАИС'' с табличной частью ``Товары'', соответствующей этим показателям. На основе показателей колонки ``Постановка / списание'' создаются документы ``Акт постановки на баланс в торговом зале ЕГАИС'', если числовое значение положительное, или ``Акт списания с баланса в торговом зале ЕГАИС'', если числовой показатель отрицательный. Все эти документы сразу создадутся после нажатия кнопки ``Перейти вперед''.
\begin{figure}[htbp]
\centering

\noindent\sphinxincludegraphics{{korost06}.png}
\end{figure}

После того как документы созданы, вы можете их отправить в ЕГАИС для корректировки данных.


\chapter{Сервисные функции ЕГАИС}
\label{servicefunctions::doc}\label{servicefunctions:id1}

\section{Запрос движений по справкам Б}
\label{servicefunctions:id2}
УТМ может возвращать движения по указанной справке.

Для этого выберите в поле ``Тип запроса'' укажите ``Запрос движений по справкам Б'' и выберите справку Б либо из списка выбора, либо вручную написав номер справки в поле.

Запишите документ и нажмите кнопку ``Отправить запрос''.
\begin{figure}[htbp]
\centering

\noindent\sphinxincludegraphics{{servicefunctions01}.png}
\end{figure}

Дождитесь ответа из ЕГАИС.

После нажатия кнопки обмена с ЕГАИС в протоколе, если ответ на запрос пришел, в табличной части документа будут показаны движения по справкам Б.
\begin{figure}[htbp]
\centering

\noindent\sphinxincludegraphics{{servicefunctions02}.png}
\end{figure}

Также вы можете по двойному нажатию на строку посмотреть документ, который сделал движение этой справки, если он есть в информационной базе.


\section{Запрос необработанных ТТН}
\label{servicefunctions:id3}


\renewcommand{\indexname}{Алфавитный указатель}
\footnotesize\raggedright\printindex
\end{document}
